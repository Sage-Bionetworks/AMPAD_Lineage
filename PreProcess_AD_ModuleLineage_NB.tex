\documentclass[]{article}
\usepackage{lmodern}
\usepackage{amssymb,amsmath}
\usepackage{ifxetex,ifluatex}
\usepackage{fixltx2e} % provides \textsubscript
\ifnum 0\ifxetex 1\fi\ifluatex 1\fi=0 % if pdftex
  \usepackage[T1]{fontenc}
  \usepackage[utf8]{inputenc}
\else % if luatex or xelatex
  \ifxetex
    \usepackage{mathspec}
  \else
    \usepackage{fontspec}
  \fi
  \defaultfontfeatures{Ligatures=TeX,Scale=MatchLowercase}
\fi
% use upquote if available, for straight quotes in verbatim environments
\IfFileExists{upquote.sty}{\usepackage{upquote}}{}
% use microtype if available
\IfFileExists{microtype.sty}{%
\usepackage{microtype}
\UseMicrotypeSet[protrusion]{basicmath} % disable protrusion for tt fonts
}{}
\usepackage[margin=1in]{geometry}
\usepackage{hyperref}
\hypersetup{unicode=true,
            pdftitle={R Notebook},
            pdfborder={0 0 0},
            breaklinks=true}
\urlstyle{same}  % don't use monospace font for urls
\usepackage{color}
\usepackage{fancyvrb}
\newcommand{\VerbBar}{|}
\newcommand{\VERB}{\Verb[commandchars=\\\{\}]}
\DefineVerbatimEnvironment{Highlighting}{Verbatim}{commandchars=\\\{\}}
% Add ',fontsize=\small' for more characters per line
\usepackage{framed}
\definecolor{shadecolor}{RGB}{248,248,248}
\newenvironment{Shaded}{\begin{snugshade}}{\end{snugshade}}
\newcommand{\KeywordTok}[1]{\textcolor[rgb]{0.13,0.29,0.53}{\textbf{#1}}}
\newcommand{\DataTypeTok}[1]{\textcolor[rgb]{0.13,0.29,0.53}{#1}}
\newcommand{\DecValTok}[1]{\textcolor[rgb]{0.00,0.00,0.81}{#1}}
\newcommand{\BaseNTok}[1]{\textcolor[rgb]{0.00,0.00,0.81}{#1}}
\newcommand{\FloatTok}[1]{\textcolor[rgb]{0.00,0.00,0.81}{#1}}
\newcommand{\ConstantTok}[1]{\textcolor[rgb]{0.00,0.00,0.00}{#1}}
\newcommand{\CharTok}[1]{\textcolor[rgb]{0.31,0.60,0.02}{#1}}
\newcommand{\SpecialCharTok}[1]{\textcolor[rgb]{0.00,0.00,0.00}{#1}}
\newcommand{\StringTok}[1]{\textcolor[rgb]{0.31,0.60,0.02}{#1}}
\newcommand{\VerbatimStringTok}[1]{\textcolor[rgb]{0.31,0.60,0.02}{#1}}
\newcommand{\SpecialStringTok}[1]{\textcolor[rgb]{0.31,0.60,0.02}{#1}}
\newcommand{\ImportTok}[1]{#1}
\newcommand{\CommentTok}[1]{\textcolor[rgb]{0.56,0.35,0.01}{\textit{#1}}}
\newcommand{\DocumentationTok}[1]{\textcolor[rgb]{0.56,0.35,0.01}{\textbf{\textit{#1}}}}
\newcommand{\AnnotationTok}[1]{\textcolor[rgb]{0.56,0.35,0.01}{\textbf{\textit{#1}}}}
\newcommand{\CommentVarTok}[1]{\textcolor[rgb]{0.56,0.35,0.01}{\textbf{\textit{#1}}}}
\newcommand{\OtherTok}[1]{\textcolor[rgb]{0.56,0.35,0.01}{#1}}
\newcommand{\FunctionTok}[1]{\textcolor[rgb]{0.00,0.00,0.00}{#1}}
\newcommand{\VariableTok}[1]{\textcolor[rgb]{0.00,0.00,0.00}{#1}}
\newcommand{\ControlFlowTok}[1]{\textcolor[rgb]{0.13,0.29,0.53}{\textbf{#1}}}
\newcommand{\OperatorTok}[1]{\textcolor[rgb]{0.81,0.36,0.00}{\textbf{#1}}}
\newcommand{\BuiltInTok}[1]{#1}
\newcommand{\ExtensionTok}[1]{#1}
\newcommand{\PreprocessorTok}[1]{\textcolor[rgb]{0.56,0.35,0.01}{\textit{#1}}}
\newcommand{\AttributeTok}[1]{\textcolor[rgb]{0.77,0.63,0.00}{#1}}
\newcommand{\RegionMarkerTok}[1]{#1}
\newcommand{\InformationTok}[1]{\textcolor[rgb]{0.56,0.35,0.01}{\textbf{\textit{#1}}}}
\newcommand{\WarningTok}[1]{\textcolor[rgb]{0.56,0.35,0.01}{\textbf{\textit{#1}}}}
\newcommand{\AlertTok}[1]{\textcolor[rgb]{0.94,0.16,0.16}{#1}}
\newcommand{\ErrorTok}[1]{\textcolor[rgb]{0.64,0.00,0.00}{\textbf{#1}}}
\newcommand{\NormalTok}[1]{#1}
\usepackage{graphicx,grffile}
\makeatletter
\def\maxwidth{\ifdim\Gin@nat@width>\linewidth\linewidth\else\Gin@nat@width\fi}
\def\maxheight{\ifdim\Gin@nat@height>\textheight\textheight\else\Gin@nat@height\fi}
\makeatother
% Scale images if necessary, so that they will not overflow the page
% margins by default, and it is still possible to overwrite the defaults
% using explicit options in \includegraphics[width, height, ...]{}
\setkeys{Gin}{width=\maxwidth,height=\maxheight,keepaspectratio}
\IfFileExists{parskip.sty}{%
\usepackage{parskip}
}{% else
\setlength{\parindent}{0pt}
\setlength{\parskip}{6pt plus 2pt minus 1pt}
}
\setlength{\emergencystretch}{3em}  % prevent overfull lines
\providecommand{\tightlist}{%
  \setlength{\itemsep}{0pt}\setlength{\parskip}{0pt}}
\setcounter{secnumdepth}{0}
% Redefines (sub)paragraphs to behave more like sections
\ifx\paragraph\undefined\else
\let\oldparagraph\paragraph
\renewcommand{\paragraph}[1]{\oldparagraph{#1}\mbox{}}
\fi
\ifx\subparagraph\undefined\else
\let\oldsubparagraph\subparagraph
\renewcommand{\subparagraph}[1]{\oldsubparagraph{#1}\mbox{}}
\fi

%%% Use protect on footnotes to avoid problems with footnotes in titles
\let\rmarkdownfootnote\footnote%
\def\footnote{\protect\rmarkdownfootnote}

%%% Change title format to be more compact
\usepackage{titling}

% Create subtitle command for use in maketitle
\newcommand{\subtitle}[1]{
  \posttitle{
    \begin{center}\large#1\end{center}
    }
}

\setlength{\droptitle}{-2em}
  \title{R Notebook}
  \pretitle{\vspace{\droptitle}\centering\huge}
  \posttitle{\par}
  \author{}
  \preauthor{}\postauthor{}
  \date{}
  \predate{}\postdate{}


\begin{document}
\maketitle

\subsection{Loading and pre-processing
data}\label{loading-and-pre-processing-data}

\begin{Shaded}
\begin{Highlighting}[]
\CommentTok{#Loading and pre-processing data}

\KeywordTok{setwd}\NormalTok{(}\StringTok{'E:/SageDocs/PredictingDriverGenes/LineageMisc/'}\NormalTok{)}

\NormalTok{Dat <-}\StringTok{ }\KeywordTok{read.delim}\NormalTok{(}\StringTok{'MAYO_CBE_TCX_logCPM.tsv'}\NormalTok{,}\DataTypeTok{stringsAsFactors =}\NormalTok{ F)}
\NormalTok{Dat2 <-}\StringTok{ }\KeywordTok{read.delim}\NormalTok{(}\StringTok{'MAYO_CBE_TCX_Covariates.tsv'}\NormalTok{,}\DataTypeTok{stringsAsFactors =}\NormalTok{ F)}

\NormalTok{AMP_mods <-}\StringTok{  }\KeywordTok{read.csv}\NormalTok{(}\StringTok{'TCX_AMPAD_Modules.csv'}\NormalTok{)}

\NormalTok{GeneNames <-}\StringTok{ }\NormalTok{Dat}\OperatorTok{$}\NormalTok{ensembl_gene_id}
\NormalTok{GeneNamesAD <-}\StringTok{ }\NormalTok{AMP_mods}\OperatorTok{$}\NormalTok{GeneID}

\NormalTok{Names <-}\StringTok{ }\KeywordTok{colnames}\NormalTok{(Dat)}

\ControlFlowTok{for}\NormalTok{ (i }\ControlFlowTok{in} \DecValTok{1}\OperatorTok{:}\KeywordTok{length}\NormalTok{(Names))\{}
  
\NormalTok{  Names[i] <-}\StringTok{ }\KeywordTok{substring}\NormalTok{(Names[i],}\DecValTok{2}\NormalTok{)}
  
\NormalTok{\}}

\KeywordTok{colnames}\NormalTok{(Dat) <-}\StringTok{ }\NormalTok{Names}
\NormalTok{cNames <-}\StringTok{ }\NormalTok{Dat2}\OperatorTok{$}\NormalTok{SampleID}
\NormalTok{l <-}\StringTok{ }\KeywordTok{length}\NormalTok{(Names)}

\CommentTok{#deleting columns not in the covariate list }
\ControlFlowTok{for}\NormalTok{ (i }\ControlFlowTok{in} \DecValTok{1}\OperatorTok{:}\NormalTok{l)\{}
  \ControlFlowTok{if}\NormalTok{ (}\OperatorTok{!}\NormalTok{(Names[i] }\OperatorTok\StringTok{ }\NormalTok{cNames))\{}
\NormalTok{    Dat[,i] <-}\StringTok{ }\OtherTok{NULL}
\NormalTok{  \}}
\NormalTok{\}}


\CommentTok{#Normalize all columns }
\KeywordTok{source}\NormalTok{(}\StringTok{'MiscPreprocessing.R'}\NormalTok{)}

\NormalTok{DatNorm <-}\StringTok{ }\KeywordTok{ColNorm}\NormalTok{(Dat)}
\NormalTok{In <-}\StringTok{ }\KeywordTok{which}\NormalTok{(GeneNames }\OperatorTok\StringTok{ }\NormalTok{GeneNamesAD)}
\NormalTok{DatNorm2 <-}\StringTok{ }\NormalTok{DatNorm[In,]}
\KeywordTok{library}\NormalTok{(Rtsne)}
\end{Highlighting}
\end{Shaded}

\begin{verbatim}
## Warning: package 'Rtsne' was built under R version 3.3.3
\end{verbatim}

\begin{Shaded}
\begin{Highlighting}[]
\NormalTok{Temp <-}\StringTok{ }\KeywordTok{Rtsne}\NormalTok{(}\KeywordTok{t}\NormalTok{(DatNorm2))}
\end{Highlighting}
\end{Shaded}

\subsection{Visualizing the whole dataset using
tSNE}\label{visualizing-the-whole-dataset-using-tsne}

\begin{Shaded}
\begin{Highlighting}[]
\CommentTok{#tSNE visualization for whole dataset}
\KeywordTok{plot}\NormalTok{(Temp}\OperatorTok{$}\NormalTok{Y[,}\DecValTok{1}\NormalTok{],Temp}\OperatorTok{$}\NormalTok{Y[,}\DecValTok{2}\NormalTok{], }\DataTypeTok{col =} \KeywordTok{as.factor}\NormalTok{(Dat2}\OperatorTok{$}\NormalTok{Sex))}
\end{Highlighting}
\end{Shaded}

\includegraphics{PreProcess_AD_ModuleLineage_NB_files/figure-latex/unnamed-chunk-2-1.pdf}

\subsection{Visualizing the one brain region using
tSNE}\label{visualizing-the-one-brain-region-using-tsne}

\begin{Shaded}
\begin{Highlighting}[]
\CommentTok{#Keeping only TCX data }
\NormalTok{In_BR <-}\StringTok{ }\KeywordTok{grep}\NormalTok{(}\StringTok{'TCX'}\NormalTok{,Dat2}\OperatorTok{$}\NormalTok{Tissue.Diagnosis)}
\NormalTok{DatNorm3 <-}\StringTok{ }\NormalTok{DatNorm2[,In_BR]}
\NormalTok{Dat3 <-}\StringTok{ }\NormalTok{Dat2[In_BR,]}
\NormalTok{Temp <-}\StringTok{ }\KeywordTok{Rtsne}\NormalTok{(}\KeywordTok{t}\NormalTok{(DatNorm3))}
\KeywordTok{plot}\NormalTok{(Temp}\OperatorTok{$}\NormalTok{Y[,}\DecValTok{1}\NormalTok{],Temp}\OperatorTok{$}\NormalTok{Y[,}\DecValTok{2}\NormalTok{], }\DataTypeTok{col =} \KeywordTok{as.factor}\NormalTok{(Dat3}\OperatorTok{$}\NormalTok{Sex))}
\end{Highlighting}
\end{Shaded}

\includegraphics{PreProcess_AD_ModuleLineage_NB_files/figure-latex/unnamed-chunk-3-1.pdf}

\subsection{Visualizing one gender using
tSNE}\label{visualizing-one-gender-using-tsne}

\begin{Shaded}
\begin{Highlighting}[]
\CommentTok{#Keeping only female data }
\NormalTok{In_S <-}\StringTok{ }\KeywordTok{grep}\NormalTok{(}\StringTok{'FEMALE'}\NormalTok{,Dat3}\OperatorTok{$}\NormalTok{Sex)}
\NormalTok{DatNorm4 <-}\StringTok{ }\NormalTok{DatNorm3[,In_S]}
\NormalTok{Dat4 <-}\StringTok{ }\NormalTok{Dat3[In_S,]}
\NormalTok{Temp <-}\StringTok{ }\KeywordTok{Rtsne}\NormalTok{(}\KeywordTok{t}\NormalTok{(DatNorm4))}
\KeywordTok{plot}\NormalTok{(Temp}\OperatorTok{$}\NormalTok{Y[,}\DecValTok{1}\NormalTok{],Temp}\OperatorTok{$}\NormalTok{Y[,}\DecValTok{2}\NormalTok{], }\DataTypeTok{col =} \KeywordTok{as.factor}\NormalTok{(Dat4}\OperatorTok{$}\NormalTok{Tissue.Diagnosis))}
\end{Highlighting}
\end{Shaded}

\includegraphics{PreProcess_AD_ModuleLineage_NB_files/figure-latex/unnamed-chunk-4-1.pdf}

\subsection{Performing lineage inference using
Monocle2}\label{performing-lineage-inference-using-monocle2}

\begin{Shaded}
\begin{Highlighting}[]
\CommentTok{#Performing lineage inference with Monocle2}
\KeywordTok{source}\NormalTok{(}\StringTok{'LineageFunctions.R'}\NormalTok{)}
\NormalTok{temp <-}\StringTok{ }\NormalTok{DatNorm4}
\KeywordTok{rownames}\NormalTok{(temp) <-}\StringTok{ }\OtherTok{NULL}
\KeywordTok{colnames}\NormalTok{(temp) <-}\StringTok{ }\OtherTok{NULL}
\NormalTok{MonRun <-}\StringTok{ }\KeywordTok{RunMonocleTobit}\NormalTok{(temp, Dat4}\OperatorTok{$}\NormalTok{AgeAtDeath)}
\end{Highlighting}
\end{Shaded}

\begin{verbatim}
## Loading required package: Matrix
\end{verbatim}

\begin{verbatim}
## Warning: package 'Matrix' was built under R version 3.3.3
\end{verbatim}

\begin{verbatim}
## Loading required package: Biobase
\end{verbatim}

\begin{verbatim}
## Loading required package: BiocGenerics
\end{verbatim}

\begin{verbatim}
## Loading required package: parallel
\end{verbatim}

\begin{verbatim}
## 
## Attaching package: 'BiocGenerics'
\end{verbatim}

\begin{verbatim}
## The following objects are masked from 'package:parallel':
## 
##     clusterApply, clusterApplyLB, clusterCall, clusterEvalQ,
##     clusterExport, clusterMap, parApply, parCapply, parLapply,
##     parLapplyLB, parRapply, parSapply, parSapplyLB
\end{verbatim}

\begin{verbatim}
## The following object is masked from 'package:Matrix':
## 
##     which
\end{verbatim}

\begin{verbatim}
## The following objects are masked from 'package:stats':
## 
##     IQR, mad, xtabs
\end{verbatim}

\begin{verbatim}
## The following objects are masked from 'package:base':
## 
##     anyDuplicated, append, as.data.frame, cbind, colnames,
##     do.call, duplicated, eval, evalq, Filter, Find, get, grep,
##     grepl, intersect, is.unsorted, lapply, lengths, Map, mapply,
##     match, mget, order, paste, pmax, pmax.int, pmin, pmin.int,
##     Position, rank, rbind, Reduce, rownames, sapply, setdiff,
##     sort, table, tapply, union, unique, unsplit, which, which.max,
##     which.min
\end{verbatim}

\begin{verbatim}
## Welcome to Bioconductor
## 
##     Vignettes contain introductory material; view with
##     'browseVignettes()'. To cite Bioconductor, see
##     'citation("Biobase")', and for packages 'citation("pkgname")'.
\end{verbatim}

\begin{verbatim}
## Loading required package: ggplot2
\end{verbatim}

\begin{verbatim}
## Warning: package 'ggplot2' was built under R version 3.3.3
\end{verbatim}

\begin{verbatim}
## Loading required package: VGAM
\end{verbatim}

\begin{verbatim}
## Warning: package 'VGAM' was built under R version 3.3.3
\end{verbatim}

\begin{verbatim}
## Loading required package: stats4
\end{verbatim}

\begin{verbatim}
## Loading required package: splines
\end{verbatim}

\begin{verbatim}
## Loading required package: DDRTree
\end{verbatim}

\begin{verbatim}
## Warning: package 'DDRTree' was built under R version 3.3.3
\end{verbatim}

\begin{verbatim}
## Loading required package: irlba
\end{verbatim}

\begin{verbatim}
## Warning: package 'irlba' was built under R version 3.3.3
\end{verbatim}

\begin{verbatim}
## Warning in newCellDataSet(as.matrix(HSMM_expr_matrix), phenoData = pd,
## featureData = fd, : Warning: featureData must contain a column verbatim
## named 'gene_short_name' for certain functions

## Warning in newCellDataSet(as.matrix(HSMM_expr_matrix), phenoData = pd,
## featureData = fd, : Warning: featureData must contain a column verbatim
## named 'gene_short_name' for certain functions

## Warning in newCellDataSet(as.matrix(HSMM_expr_matrix), phenoData = pd,
## featureData = fd, : Warning: featureData must contain a column verbatim
## named 'gene_short_name' for certain functions
\end{verbatim}

\begin{verbatim}
## Warning: package 'bindrcpp' was built under R version 3.3.3
\end{verbatim}

\subsection{Visualizing using Monocle's
visualization}\label{visualizing-using-monocles-visualization}

\begin{Shaded}
\begin{Highlighting}[]
\KeywordTok{plot_cell_trajectory}\NormalTok{(MonRun, }\DataTypeTok{color_by =} \StringTok{"Labels"}\NormalTok{)}
\end{Highlighting}
\end{Shaded}

\includegraphics{PreProcess_AD_ModuleLineage_NB_files/figure-latex/unnamed-chunk-6-1.pdf}

\subsection{Visualizing Monocle2 with different
labels}\label{visualizing-monocle2-with-different-labels}

\begin{Shaded}
\begin{Highlighting}[]
\KeywordTok{Monocle.Plot}\NormalTok{(MonRun, }\DataTypeTok{Labels =}\NormalTok{ Dat4}\OperatorTok{$}\NormalTok{Tissue.Diagnosis, }\DataTypeTok{Discrete =}\NormalTok{ T)}
\end{Highlighting}
\end{Shaded}

\begin{verbatim}
## Warning: package 'reshape2' was built under R version 3.3.3
\end{verbatim}

\includegraphics{PreProcess_AD_ModuleLineage_NB_files/figure-latex/unnamed-chunk-7-1.pdf}

\subsection{Viewing gene expression overlaid on the
lineage}\label{viewing-gene-expression-overlaid-on-the-lineage}

\begin{Shaded}
\begin{Highlighting}[]
\NormalTok{gList <-}\StringTok{ }\NormalTok{GeneNames[In]}
\NormalTok{g <-}\StringTok{ }\NormalTok{GeneNames[In[}\DecValTok{1}\NormalTok{]]}
  
\KeywordTok{Mon.Plot.Genes}\NormalTok{(MonRun, DatNorm4, gList, g)}
\end{Highlighting}
\end{Shaded}

\includegraphics{PreProcess_AD_ModuleLineage_NB_files/figure-latex/unnamed-chunk-8-1.pdf}


\end{document}
